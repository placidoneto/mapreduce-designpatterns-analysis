For performing our systematic mapping we adopt the method proposed in \cite{SM} consisting  of five  steps: 
\begin{enumerate}
\item \textbf{Definition of research questions}, to determine the research
scope;  
\item \textbf{Search of primary papers}, to select candidate papers expressing a
query for retrieving references from scientific databases; 
\item \textbf{Screening of papers}, to identify relevant papers using
inclusion and exclusion criteria   to narrow
the number of papers of interest; 
\item \textbf{Keywording of abstracts}, to identify terms that are used
for producing  classification schemes (mapping categories); 
\item \textbf{Data extraction and mapping process}, to produce the systematic
mapping by clustering the papers into the mapping categories.
\end{enumerate}

The results of the systematic mapping are presented as
bubble plots in the next section. These visual representations have two main
benefits: (i) it can show the relevance of the mapping facets/categories
has changed over time and (ii) it can reveal research opportunities
hidden on the blanks found among classification categories, pointing
topic areas and types of publications yet to be investigated.

%.........................................................
\subsection{Research questions}
\label{sec:ResearchQuestions}
%.........................................................

... In order to achieve this goal we formulated three research questions:

\begin{itemize}
\item {\em RQ1: Which are the main map reduce design patterns used in data
analysis?} This question is devoted to measure the map reduce design patterns
used in big data analysis and to verify which of them are the most common.

\item {\em RQ2: What type of solutions have been proposed considering map
reduce design patterns?} This question will help us to identify how map reduce
design patterns are used in data analysis, and also
what type of solution have been proposed for each kind of pattern or
composition of patterns. It will let us identify how the patterns can be
composed to a better big data analysis result.
 
\item  {\em RQ3: Which type of relationship between patterns are proposed to
improve the results of big data analysis?} The map reduce design patterns are
classified considering and addressing different aspects
according big data domain. There is a list o patterns organized into categories
such as summarization, filtering, data organization and join patterns. These
patterns can be composed in order to get a better result of the big data
analysis. Thus, this research question aims to verify if the patterns are often
compounds, and which kind of composition is made.


\end{itemize}

%.........................................................
\subsection{Search and Retrieval of Papers}

%\subsection{Expressing the  query for generating a data collection}
%.........................................................
Considering the research questions, we defined a set of keywords to be used for
searching relevant works. Based on these keywords and their
correlated words the query used was:        
 
\begin{quote} \sl
\qquad  (big data \textbf{OR} bigdata \textbf{OR} map reduce \textbf{OR}
map-reduce \textbf{OR} mapreduce \textbf{OR} hadoop \textbf{OR} pig)

\textbf{AND}

\qquad (design patterns \textbf{OR} designpatterns \textbf{OR}
design-patterns \textbf{OR} design pattern
\textbf{OR} designpattern \textbf{OR} design-pattern)
\end{quote}

% 
% \begin{table}\centering
% \begin{tabular}{|l|c|c|c|} \hline
% \textbf{Source/Action}	& \textbf{Included}	& \textbf{Excluded}	& \textbf{Total}	\\ \hline
% \textit{ACM-DL}				& 3								& 3								& 6						\\ \hline
% \textit{IEEE}						& 28								& 156								& 188					\\ \hline
% \textit{Science Direct}	& 1								& 1							& 2					\\ \hline
% \textbf{Total}					& \textbf{32}			& \textbf{164}			& \textbf{196}	\\ \hline
% \end{tabular}
% \caption{\label{table:Sources}Sources and number of papers.}
% \end{table}

We searched and filtered relevant works in four steps. 
In the first  step we
searched in four databases: \textit{Science Direct}\footnote{\tt
http://www.sciencedirect.com/}, \textit{IEEE}\footnote{\tt
http://ieeexplore.ieee.org/} and \textit{ACM Digital Library}\footnote{\tt
http://dl.acm.org/}. %(see table \ref{table:Sources})
We retrieved 196 works. 
The search was done for relevant publications from 1998 to 2014. 
We stored in several spreadsheets (one per database) the title, year,  and abstract of each reference we found. 

In the second step, we perform a data cleaning by excluding repeated works.  

We performed another filtering procedure by screening the title and the abstract
of the papers, looking for those papers that are relevant to our study.
In this process, we excluded 164 works. 
% The columns \textbf{Included} and \textbf{Excluded} in Table~\ref{table:Sources}
% show the number of papers that were considered (resp. excluded) on our study.

Finally, in the last step we built the final data collection using exclusion
and inclusion criteria shown in table \ref{table:criteria}. The final data
collection contained 32 papers.

 
\begin{table}\centering \small
\begin{tabular}{|l|} \hline
\textbf{Inclusion criteria}		\\ \hline\hline
- Text in English								\\ \hline
- Peer reviewed journals, conferences or workshops	\\ \hline
- PhD and master thesis	\\ \hline
- Focus on Map Reduce Design Patter and Big Data				\\ \hline\hline
\textbf{Exclusion criteria}		\\ \hline\hline	
- Abstracts, tutorials, short papers, PhD workshops, demonstrations, technical reports		\\ \hline	
- Papers dealing with service service lookup		\\ \hline
- Papers dealing with service matching 	\\ \hline
\end{tabular}
\caption{\label{table:criteria} Inclusion and exclusion criteria.}
\end{table}
