\documentclass[preprint,12pt]{elsarticle}
\usepackage{geometry}
%\geometry{letterpaper}                   % ... or a4paper or a5paper or ...
\usepackage{graphicx}
\usepackage{xspace}
\usepackage{amssymb} 
\usepackage{epstopdf}

 
\usepackage{datatool}
\usepackage{tikz}
\usepackage{pgfplots}
\usepackage{pgfplotstable}
\usetikzlibrary{patterns}
\usepackage{lscape} 
\usepackage{subfig}
\usepackage{multirow}
\usepackage{rotating}


%% Use the option review to obtain double line spacing
%% \documentclass[preprint,review,12pt]{elsarticle}
   
%% Use the options 1p,twocolumn; 3p; 3p,twocolumn; 5p; or 5p,twocolumn
%% for a journal layout: 
%% \documentclass[final,1p,times]{elsarticle}
%% \documentclass[final,1p,times,twocolumn]{elsarticle}
%% \documentclass[final,3p,times]{elsarticle}
%% \documentclass[final,3p,times,twocolumn]{elsarticle}
%% \documentclass[final,5p,times]{elsarticle}
%% \documentclass[final,5p,times,twocolumn]{elsarticle}

%% if you use PostScript figures in your article
%% use the graphics package for simple commands
%% \usepackage{graphics}
%% or use the graphicx package for more complicated commands
%% \usepackage{graphicx}
%% or use the epsfig package if you prefer to use the old commands
%% \usepackage{epsfig}

%% The amssymb package provides various useful mathematical symbols
\usepackage{amssymb}
%% The amsthm package provides extended theorem environments
%% \usepackage{amsthm}

%% The lineno packages adds line numbers. Start line numbering with
%% \begin{linenumbers}, end it with \end{linenumbers}. Or switch it on
%% for the whole article with \linenumbers after \end{frontmatter}.
%% \usepackage{lineno}

%% natbib.sty is loaded by default. However, natbib options can be
%% provided with \biboptions{...} command. Following options are
%% valid:

%%   round  -  round parentheses are used (default)
%%   square -  square brackets are used   [option]
%%   curly  -  curly braces are used      {option}
%%   angle  -  angle brackets are used    <option>
%%   semicolon  -  multiple citations separated by semi-colon
%%   colon  - same as semicolon, an earlier confusion
%%   comma  -  separated by comma
%%   numbers-  selects numerical citations
%%   super  -  numerical citations as superscripts
%%   sort   -  sorts multiple citations according to order in ref. list
%%   sort&compress   -  like sort, but also compresses numerical citations
%%   compress - compresses without sorting
%%
%% \biboptions{comma,round}

% \biboptions{}


\journal{Journal of Systems and Software}

%%% OUR MACROS %%%
\newcommand{\COMMENT}[1]{ }

%\usepackage[usenames,dvipsnames]{xcolor}
\usepackage{xcolor}


\usepackage{amsmath}
\usepackage[thmmarks,amsmath]{ntheorem}

\newcommand{\openbox}{\leavevmode
  \hbox to.77778em{%
  \hfil\vrule
  \vbox to.675em{\hrule width.6em\vfil\hrule}%
  \vrule\hfil}}

\theoremstyle{plain}
\theoremheaderfont{\normalfont\bfseries}
\theorembodyfont{\normalfont}
\theoremseparator{}
\theoremindent0cm
\theoremnumbering{arabic}
\newtheorem{algo}{Algorithm}

\theoremstyle{plain}
%\theoremheaderfont{\normalfont\itshape}
\theoremheaderfont{\normalfont\bfseries}
\theorembodyfont{\normalfont}
\theoremseparator{}
\theoremindent0cm
\theoremnumbering{arabic}
\theoremsymbol{\ensuremath{\openbox}} 
\newtheorem{example}{Example}


\theoremstyle{plain}
\theoremheaderfont{\normalfont\bfseries}
\theorembodyfont{\normalfont}
\theoremseparator{.}
\theoremindent0cm
\theoremnumbering{arabic}
\theoremsymbol{\ensuremath{\Box}} 
\newtheorem{defi}{Definition}

\theoremstyle{plain} 
\theoremsymbol{\ensuremath{\Box}} 
\theoremseparator{.} 
\newtheorem{prop}{Property}

\def\FlyingPig{\textsl{FlyingPig}}

\newcounter{numberInTrivlist}

\newenvironment{numtrivlist}{\begin{list}{\rm \arabic{numberInTrivlist})} 
                                         {\usecounter{numberInTrivlist}
                                          \setlength{\leftmargin}{0pt}
                                          \setlength{\rightmargin}{0pt}
                                          \setlength{\itemindent}{12pt}
                                          \setlength{\listparindent}{0pt}}}
                            {\end{list}}

\newenvironment{itemizedTrivlist}{\begin{list}{\rm ~\hspace{2mm} $\bullet$\ } 
                                         {\setlength{\leftmargin}{0pt}
                                          \setlength{\rightmargin}{0pt}
                                          \setlength{\itemindent}{12pt}
                                          \setlength{\listparindent}{0pt}}}
                            {\end{list}}

\usepackage{listings}


\lstset{numbers=right, numbersep=5pt, numberstyle=\tiny, stepnumber=1,escapechar=\!,columns=fullflexible,
        morekeywords={procedure,let,for,do,if,then,else,add,choose,end,while,
        true,false,rise,exception,extend,resume,to,return,function}}

\newcommand{\pisodm}[0]{$\pi$SOD-M\xspace}

\begin{document}

\begin{frontmatter}

%% Title, authors and addresses

%% use the tnoteref command within \title for footnotes;
%% use the tnotetext command for the associated footnote;
%% use the fnref command within \author or \address for footnotes;
%% use the fntext command for the associated footnote;
%% use the corref command within \author for corresponding author footnotes;
%% use the cortext command for the associated footnote;
%% use the ead command for the email address,
%% and the form \ead[url] for the home page:
%%
%% \title{Title\tnoteref{label1}}
%% \tnotetext[label1]{}
%% \author{Name\corref{cor1}\fnref{label2}}
%% \ead{email address}
%% \ead[url]{home page}
%% \fntext[label2]{}
%% \cortext[cor1]{}
%% \address{Address\fnref{label3}}
%% \fntext[label3]{}

\title{Analyzing map reduce design patterns}

%% use optional labels to link authors explicitly to addresses:
%% \author[label1,label2]{<author name>}
%% \address[label1]{<address>}
%% \address[label2]{<address>}




\author[inst1]{Pl\'acido A. Souza Neto}
\author[inst2]{Genoveva Vargas-Solar}

 
\address[inst1]{Instituto Federal do Rio Grande do Norte -- Natal, Brazil}
\address[inst2]{CNRS, LIG-LAFMIA, Saint Martin d'H\`eres, France}


\begin{abstract}

This paper discusses the results ...

\end{abstract}

\begin{keyword}
%% keywords here, in the form: keyword \sep keyword
Map reduce \sep Design patterns \sep Analysis \sep Big data.

%% MSC codes here, in the form: \MSC code \sep code
%% or \MSC[2008] code \sep code (2000 is the default)

\end{keyword}

\end{frontmatter}

%%
%% Start line numbering here if you want
%%
% \linenumbers

%% main text
%*********************************************************************************************************
\section{Introduction}

\section{Mapping process}\label{sec:mappingprocess}


%.........................................................
\subsection{Search and Retrieval of Papers}

%\subsection{Expressing the  query for generating a data collection}
%.........................................................
Considering the research questions, we defined a set of keywords to be used for
searching relevant works. Based on these keywords and their
correlated words the query used was:        
 
\begin{quote} \sl
\qquad  (big data \textbf{OR} bigdata \textbf{OR} map reduce \textbf{OR}
map-reduce \textbf{OR} mapreduce \textbf{OR} hadoop \textbf{OR} pig)

\textbf{AND}

\qquad (design patterns \textbf{OR} designpatterns \textbf{OR}
design-patterns \textbf{OR} design pattern
\textbf{OR} designpattern \textbf{OR} design-pattern)
\end{quote} 

%% References with bibTeX database:

\bibliographystyle{plain}
\bibliography{bibliography}


\end{document}

%%
%% End of file `elsarticle-template-1a-num.tex'.
