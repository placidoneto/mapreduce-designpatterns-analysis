\subsection{Political Big Data Analysis}
\label{:sec:election-analysis}

% Big data and political event  (both, a political event or an
% election) can be used considering what users react publishing their point of
% view in social networks, by analysing online information seeking during the
% political event.
% 
% The overall aim of this case study is to enhance current research on the use of
% internet generated "big data" coming social media for
% understand how people react considering a political event.
% This is an area which has generated considerable interest in computer science and political science, yet
% also much disappointment, following the poor predictive power identified by much
% current research (for example, the number of times vips - very importante
% persons/politicals - are mentioned on Twitter).

The analysis is done considering the map reduce design patterns presented in
\cite{}. The execution values is presented for 5 data size intervals: 1-5Mb;
50-60Mb; 90-120Mb; 450-550Mb; and 900Mb-1.1Gb.
 
\subsection{Summarization Pattern} 
 
\textit{Problem-1:} Given a list of user’s comments, determine the first and
last time a user commented and the total number of comments from that user. 

\textbf{done. numerical.pig file}
             
\textit{Problem-2:} Given a list of user’s comments, determine the average comment length per
hour of day.

\textbf{not done yet. Problem in defining mediana e standar deviation.}
             

\textit{Problem-2:} Given a set of user’s comments, build an inverted index of
Wikipedia URLs to a set of answer post IDs .

\subsubsection{Memory Management}

Pig allocates a fix amount of memory to store bags and spills to disk as soon as
the memory limit is reached. This is very similar to how Hadoop decides when to
spill data accumulated by the combiner. The amount of memory allocated to the
use of bags. The default is set to 20\% (0.2) of available memory. Note that
this memory is shared across all large bags used by the application.    
 